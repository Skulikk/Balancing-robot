% Tento soubor nahraďte vlastním souborem s přílohami (nadpisy níže jsou pouze pro příklad)

% Pro kompilaci po částech (viz projekt.tex), nutno odkomentovat a upravit
%\documentclass[../projekt.tex]{subfiles}
%\begin{document}

% Umístění obsahu paměťového média do příloh je vhodné konzultovat s vedoucím
%\chapter{Obsah přiloženého paměťového média}

\chapter{Manuál}
Před spuštěním je vhodné vizuálně zkontrolovat, jestli se z řídící desky neodpojil některý z konektorů. Následně se přepnutím zeleného tlačítka na boku robot spustí. Zatímco se načítá operační systém, je vhodné nasadit kolo a nárazníky (pokud již nejsou nasazené). Balancující program se spustí automaticky -- uživatel je na to upozorněn krátkým zvukovým signálem. V tuto chvíli je potřeba k robotovi připojit již dříve spárovaný mobilní telefon, který má nainstalovanou aplikaci Blue Dot. 

Po připojení k robotovi se v aplikaci objeví ovládací obrazovka. Pokud některá ze stavových kontrolek senzorů není zelená, značí to chybu. Nejprve je nutné zkontrolovat správné zapojení daného senzoru. Pokud je zde vše v pořádku, je potřeba program restartovat (černé tlačítko ve druhé řadě).

\chapter{Obsah paměťového média}
Přiložené paměťové médium obsahuje následující složky a soubory:
\begin{itemize}
  \item \textbf{robot} -- zdrojové soubory, zkompilované kódy projektu, použité knihovny
  \item \textbf{thesis-src} -- zdrojové soubory textu závěrečné práce
  \item \textbf{thesis.pdf} -- vysázená závěrečná práce
  \item \textbf{demonstrace projektu.mp4} -- video demonstrující fungování robota
  \item \textbf{plakát.pdf}
  \item \textbf{readme.md}
\end{itemize}

%\chapter{RelaxNG Schéma konfiguračního souboru}

%\chapter{Plakát}

% Pro kompilaci po částech (viz projekt.tex) nutno odkomentovat
%\end{document}
